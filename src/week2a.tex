

\title[Systems Engineering]{ Systems Integration  } 



\begin{frame}
\frametitle{Systems Engineering and Integration}
\begin{block}{Integration  }
Systems Engineering and Integration (SE $\& $ I) is a disciplined approach for the definition, implementation, integration and operations of a system (product or service). 

\end{block}
\end{frame}



\begin{frame}
\frametitle{Interface relationships}
\begin{block}{Integration  }

\begin{enumerate}
    \item   Emphasis is on the satisfaction of stakeholder functional, physical and operational performance requirements in the intended use environments over its planned life cycle within cost and schedule constraints.
    \item SE $\& $ I includes the engineering activities and technical management activities related to the above definition considering the interface relationships across all elements of the system, other systems or as a part of a larger system.
\end{enumerate}

 
\end{block}
\end{frame}

\begin{frame}
\frametitle{Interface with Defined Outcome }
\begin{block}{SET CLEAR GOALS}

\begin{enumerate}
    \item  A defined outcome in measurable terms
   \item  Clear direction for the program against which decisions can be made
    \item  An effective means of communicating the purpose of the endeavour to all involved
\end{enumerate}

\end{block}
\end{frame}





\begin{frame}
\frametitle{Interface with  GOVERNANCE }
\begin{block}{ESTABLISH GOVERNANCE}

\begin{enumerate}
    \item All decisions required for introducing the system into operational services will be allocated properly, with consideration of the technical, operational and programmatic risks
   \item  A framework to support assurance processes
\end{enumerate}

\end{block}
\end{frame}





\begin{frame}
\frametitle{Interface with SYSTEM MIGRATION }
\begin{block}{DEVELOP A SYSTEM MIGRATION PLAN}

\begin{enumerate}
    \item  A key planning tool which communicates the journey to the common end goal
   \item  A tool to support coordination across the program, contracts and stakeholders 
\end{enumerate}

\end{block}
\end{frame}



\begin{frame}
\frametitle{Interface with  SYSTEM ARCHITECTURE }
\begin{block}{DEVELOP THE SYSTEM ARCHITECTURE}

\begin{enumerate}
    \item  Visibility of the system concept to senior management and discipline engineers/designers and other stakeholders
   \item  Optimization of the solution at a level of detail appropriate to the phase of the program
    \item Assessment of feasibility of the solutions, highlighting risks and requirements
\end{enumerate}

\end{block}
\end{frame}




\begin{frame}
\frametitle{Interface with ENGINEERING Resources }
\begin{block}{ALIGN OPERATIONS AND ENGINEERING}

\begin{enumerate}
    \item Clear operational requirements against which the system design can be developed and verified
   \item Confidence that a solution will be developed which can be operated/maintained
\end{enumerate}

\end{block}
\end{frame}

\begin{frame}
\frametitle{Interface with Risk Analysis }
\begin{block}{IDENTIFY AND MITIGATE WHOLE-SYSTEM RISKS}

\begin{enumerate}
    \item  Better understanding of the technical uncertainty within the program, through a holistic analysis
   \item  Ability to respond rapidly to escalations of risks and issues
    \item Increased accuracy in cost and schedule forecasting
\end{enumerate}

\end{block}
\end{frame}


\begin{frame}
\frametitle{Systems Engineering and Integration}
\begin{block}{SE $\& $ I Branch   }
 The goal of the SE $\& $ I Branch is to provide systems engineering services to Programs and Projects to ensure that Required systems are designed, built, and operated in the most cost-effective way possible. 
\end{block}
\end{frame}


\begin{frame}
\frametitle{Systems Engineering and Integration}
\begin{block}{Programs and Projects  }

\begin{enumerate}
    \item  Requires high-quality systems engineering expertise, and supporting products and services required of successful Programs and Projects.
 \end{enumerate}

\end{block}
\end{frame}



\begin{frame}
\frametitle{Systems Engineering and Integration}
\begin{block}{Integration  }

\begin{enumerate}
    \item  Design specifications
    \item  Interface control
    \item Decomposition allocation and tracking
    \item  Verification tracking
    \item  Planning and process implementation
       \item  Risk analysis
       \item  Mitigation strategies
        \item  Tracking and trending
\end{enumerate}
\url{https://www.nasa.gov/centers/armstrong/capabilities/CodeR/flight/systems_engineering.html}

\end{block}
\end{frame}


\begin{frame}
\frametitle{Tracking and Trending}
\begin{block}{DELIVER INTEGRATION  }

 Building on the first two tiers within the framework,
 
\begin{enumerate}
    \item   effort now focuses on applying robust Systems Engineering (SE) techniques 
    \item   leveraging the established management process to deliver the desired outcomes.
 \end{enumerate}

\end{block}
\end{frame}

\begin{frame}
\frametitle{Extend Modeling Functionality with Custom Blocks}
\begin{block}{Customize it for Project  }


 
\begin{enumerate}
    \item   Implement new algorithms in Simulink® using MATLAB®, C/C++, and Fortran 
    \item   Simulink allows you to implement algorithms designed in MATLAB, C/C++, and Fortran. As a result, you can keep the existing functionality of your code while building more advanced models in Simulink
    
 \end{enumerate}

 \url{https://in.mathworks.com/help/simulink/implement-new-algorithm.html?s_tid=CRUX_lftnav}

\end{block}
\end{frame}

\begin{frame}
\frametitle{ARINC: The Ultimate Guide To Modern Avionics Protocol}
\begin{block}{ ARINC 429 }

The ARINC 429 defines basic requirements for the transmission of digital data between commercial avionics systems. For ease of design implementation and data communications, signal levels, timing, and protocol characteristics are specified. ARINC 429 is designed to provide commercial aircraft Line Replaceable Units (LRUs). As in a layman’s term, the ARINC protocol is designed to support communication in the Local Area Network (LAN) of the avionics.

 \url{https://www.logic-fruit.com/blog/arinc-standard/arinc/}

\end{block}
\end{frame}




\begin{frame}
\frametitle{Model-Based Engineering for Cybersecurity}

\begin{block}{ISO/SAE-21434  }
Engineering projects requires to consider the increasingly significant needs and constraints regarding expected behaviors, services, quality and security. 

These requirements are introduced into system and software engineering projects as functional and non-functional properties. Satisfying such properties implies rigorous processes that steer the project, from the requirements identification and definition to the system deployment and maintenance. 

\url{https://intercax.com/2020/02/20/mbse-and-integration-part-1/}


\end{block}
\end{frame}



\newpage 

\title[Systems Engineering]{ System Verification and Validation  } 



\begin{frame}
\frametitle{Engineering teams use Model-Based Design 
 \\ 19 May 23 }
\begin{block}{testing   }

MathWorks tools use simulation testing and formal methods-based static analysis to complement Model-Based Design with rigor and automation to find errors earlier and achieve higher quality.

\url{https://in.mathworks.com/solutions/verification-validation.html}

\end{block}
\end{frame}

\newpage 

\begin{frame}
\frametitle{Engineering teams use Model-Based Design }
\begin{block}{ Measure by using Requirements sheet}

\begin{enumerate}
    \item Trace requirements to architecture, design, tests, and code
    \item  Prove that your design meets requirements and is free of critical run-time errors
    \item Check compliance and measure quality of models and code
    \item Generate test cases automatically to increase test coverage
    \item  Produce reports and artifacts, and certify to standards (such as DO-178 and ISO 26262).
\end{enumerate}

\end{block}
\end{frame}


\newpage 

\begin{frame}
\frametitle{Engineering teams use Model-Based Design }
\begin{block}{ Requirements Verification}
\begin{enumerate}
    \item Informal text requirements that can be stored in documents, spreadsheets, or in requirements management tools, such as IBM® Rational® DOORS®, initiate the development process. 
\item  Requirements Toolbox™ allows you to import, view, author, and manage requirements together with your architecture, designs, generated code, and test artifacts. 
\end{enumerate}
\end{block}
\end{frame}


\newpage 

\begin{frame}
\frametitle{Engineering teams use Model-Based Design }
\begin{block}{ Requirements Verification}

\begin{enumerate}
    \item You can create a digital thread from requirements to design in System Composer™, Simulink, Stateflow®, or tests in Simulink Test™. 

 \item  With this traceability, you can identify implementation or testing gaps and quickly understand the impact of a change on the design or test. 
\end{enumerate}

\end{block}
\end{frame}


\newpage 

\begin{frame}
\frametitle{Engineering teams use Model-Based Design }
\begin{block}{ Requirements Verification}

\begin{enumerate}
    \item You can formalize requirements and analyze them for consistency, completeness, and correctness to validate them earlier using the Requirements Table block in Requirements Toolbox. 
 \item  With Temporal assessments in Simulink Test, you can verify text requirements by specifying assessments with precise semantics in a natural language format that can be evaluated and debugged.
\end{enumerate}

\end{block}
\end{frame}

\newpage 

\begin{frame}
\frametitle{Engineering teams use Model-Based Design }
\begin{block}{ Formalize and Validate Requirements}

\begin{enumerate}
    \item Incomplete and inconsistent requirements cause errors in the design phase that become exponentially more expensive to fix over time. Engineers can save time and money by formalizing requirements to validate them earlier. Formal requirements are expressed mathematically as unambiguous specifications that you execute through simulation to validate that they are complete, correct, and consistent.
 \item  different styles of formal requirements, how to model complete and consistent formal requirements, and how to go from stakeholder needs to formal requirements using use case diagrams and simulation.


\end{enumerate}

\end{block}
\end{frame}

\newpage 

\begin{frame}
\frametitle{Engineering teams use Model-Based Design }
\begin{block}{ Formalize and Validate Requirements}

\begin{enumerate}
    \item  Express requirements that are consistent and complete
 \item   Go from stakeholder needs to formal requirements using use case diagrams
  \item Simulate requirements to validate required model behavior 
\end{enumerate}

\end{block}
\end{frame}


\newpage 

\begin{frame}
\frametitle{Engineering teams use Model-Based Design }
\begin{block}{ Automating Verification and Validation with Simulink}

\url{https://in.mathworks.com/campaigns/offers/automating-verification-and-validation.html?s_tid=vid_pers_ofr_recs}

\end{block}
\end{frame}


\newpage 

\begin{frame}
\frametitle{Engineering teams use Model-Based Design }
\begin{block}{ MathWorks' Model Verification  and Validation tools}

\url{https://in.mathworks.com/matlabcentral/fileexchange/71399-requirements-and-advanced-model-checks-getting-started}
\end{block}
\end{frame}


\newpage 

\begin{frame}
\frametitle{Engineering teams use Model-Based Design }
\begin{block}{ Advanced Model Checks Workflow Example}

\url{https://github.com/mathworks/requirements-advanced-checks-getting-started}

\end{block}
\end{frame}


\newpage 

\begin{frame}
\frametitle{Engineering teams use Model-Based Design }
\begin{block}{Simulink Design Verifier}

\url{https://in.mathworks.com/products/simulink-design-verifier.html?s_tid=FX_PR_info}

\url{https://in.mathworks.com/videos/how-to-model-complete-and-consistent-requirements-1642669397372.html}

\end{block}
\end{frame}