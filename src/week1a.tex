
\title[Systems Engineering]{System Requirements } 

\begin{frame}
\frametitle{Overview} 
\end{frame}

%---------------
%	PRESENTATION SLIDES
%--------------

\section{System Requirements to Architecture  } 

\subsection{Descriptive Architecture Model}

\subsection{Detailed Implementation  Model}

\section{ System  Design  }

\subsection{Implementable Model} 

\subsection{ Software for Embedded system} 


\begin{frame}
\frametitle{System Engineering }
System Engineering is using mathematical science to formulate and Model a given physical process.  Deployment of obtained model in real world requires infrastructure and associated challenges.  Though Model is essential starting point in system engineering, what makes model to be useful is the kind of Optimization is used on a given model to arrive feasible model which can go for deployment and use it for service.  Mentioned optimization is key for successful product or service in industry. In the case of Research and development, every engineering segment has its own challenges to obtain model for a physical process. In this course provides balance between components such as Model, optimization and deployment. 
\end{frame}

%------------------------------------------------

\begin{frame}
\frametitle{Converging to Model}
\begin{itemize}
\item Impact and Conversation Analysis
\item Capture System Requirements
\item Optimal Architecture and Simulation ( Python, Matlab)
\item Create Simplified Model
\item Use Simulation to Verify System Architecture
\item Check for Engineering Components
\item Refine  System Architecture
\item  Simulation by using model-based systems engineering (MBSE )
\end{itemize}
\end{frame}

%------------------------------------------------


\begin{frame}
\frametitle{Tool Set to Capture System Requirements }
\begin{block}{ Tool Set}
\begin{enumerate}
 \setlength{\itemsep}{0pt}
    \item  MATLAB®
    \item  Simulink®
    \item  System Composer™ 
    \item  Requirements Toolbox™
\end{enumerate}

\url{https://in.mathworks.com/products/requirements-toolbox.html}
\url{https://in.mathworks.com/help/slrequirements/ug/generate-requirements-reports.html}
\newline 

Requirements Tool Box
\newline 
\url{https://in.mathworks.com/help/slrequirements/index.html?s_tid=CRUX_lftnav}
\url{https://youtu.be/J_y2I09rj_I}
\end{block}
\end{frame}

\begin{frame}
\frametitle{Capture System Requirements }

\begin{block}{Importing Requirements from external sources }
Requirements Toolbox provides option to import requirements from external sources.  Development environment can use imported requirements.   Using Industry standard interchange format.
\end{block}

\begin{block}{Validation of Requirements}
Formalize requirements \\
Validate requirements 
\end{block}

\begin{block}{Directly Manage Requirements}
Simulink®  and 
System Composer™  are used to capture requirements directly.
\end{block}
\end{frame}


\begin{frame}
\begin{block}{Requirements from external sources}
Requirements toolbox lets you author, link, and validate requirements within MATLAB or Simulink, track their implementation and verification status, and quickly respond to requirements changes. Requirements from external sources can be imported and viewed within your development environment. You can exchange data with other tools, using an industry standard interchange format.
\end{block}
\end{frame}

\begin{frame}
\begin{block}{Traceability links}
Navigate directly to related design elements with traceability links. If you select a requirement, it's related design elements are highlighted. And if you click a block, the linked requirements come into focus. You can also link requirements to test cases to track their verification status. The implementation and verification statuses show the completeness of designs and related tests.
\end{block}
\end{frame}


\begin{frame}
\begin{block}{ Use of Simulink and System Composer}
You can formalize requirements, with the requirements block, and validate they are correct. You can manage requirements directly within Simulink, System Composer, and Stateflow. Traceability to designs can be quickly created through drag and drop. Badges indicate where links exist in the model. And you can show requirements content directly in the diagram, without modifying the model.
\end{block}
\end{frame}


\begin{frame}
\frametitle{Traceability of  Requirements }

\begin{block}{drag and drop }
to trace requirements changes 
\end{block}

\begin{block}{Impact on change}
change in requirements and traceability diagram
\end{block}
\end{frame}

\begin{frame}
\begin{block}{Traceability matrix}
To Review traceability between artifacts, use the traceability matrix. You can directly create links where there are gaps. When a linked requirement or test changes, you are notified, so that you can determine the impact of the change. You can visualize upstream and downstream artifacts using the traceability diagram.
\end{block}
\end{frame}
%------------------------------------------------

\begin{frame}
\frametitle{Understand Impact of any changes }
\begin{columns}[c] 
\column{.45\textwidth} % Left column and width
\textbf{Requirements}
\begin{enumerate}
\item MATLAB code
\item Design
\item Generated Code
\item Simulation
\item Test Cases
\end{enumerate}

\column{.5\textwidth} % Right column and width
Establishing full lifecycle traceability enables you to document your requirements are being met, understand the impact of any changes, and create a digital thread between requirements, MATLAB code, designs, generated code, and test cases. To learn more or request a trial, use the links below.
\end{columns}
\end{frame}

%------------------------------------------------
\section{ Optimal Design}
%------------------------------------------------

%%jk\begin{frame}
%%jk\frametitle{Theorem}
%%jk\begin{theorem}[Mass--energy equivalence]
%%jk$E = mc^2$
%%jk\end{theorem}
%%jk\end{frame}



\begin{frame}
\frametitle{References}
\footnotesize{
\begin{thebibliography}{99} 
\bibitem[matlab1]{p1} MATLAB
\newblock System Engineering: From Requirements to Architecture to Simulation
\newblock \url{https://in.mathworks.com/campaigns/offers/model-based-system-engineering.html} 
\newblock Requirements 
\newblock  \url{https://youtu.be/Iblo2Il-pOA}
\end{thebibliography}
}
\end{frame}
