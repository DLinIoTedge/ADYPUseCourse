
\title[Systems Engineering]{System Acquisition and Management } 

\newpage
\begin{frame}
\frametitle{ System Acquisition and Management }
\begin{block}{System Definition-Enabled Acquisition}


A Concept for Defining Requirements for Applying Model-Based Systems Engineering (MBSE) to the Acquisition of  Complex System

\end{block}
\end{frame}


\newpage
\begin{frame}
\frametitle{ System Acquisition and Management }
\begin{block}{System Definition-Enabled Acquisition}

\begin{enumerate}
    \item International Council on Systems Engineering (INCOSE) 
    \item Object Management Group (OMG) ( UML team )
    \item SysML  2001
     \item SysML  2006  ( 1st Proposal Document )
     \item SysML 2007  V. 1. 0
      \item SysML 2018  V. 2
      \item  SysML models are designed to be exchanged using the XML Metadata Interchange (XMI) standard. 
    \item Requirements ( added in SysML)
    \begin{enumerate}
         \item  Formalised version of Requirements  ( Excel spreadsheet, IBM DOORS , SAP IPD etc ) used for dependency associations
         \item SysML specifications define the requirements modelling ( tool set is from open source or from particular company)

    \end{enumerate}
   
\end{enumerate}


\end{block}
\end{frame}


\newpage
\begin{frame}
\frametitle{ System Acquisition and Management }
\begin{block}{Visual and Graphical }

SysML define a visual and graphical representation of
textual requirements, specialised associations between themselves or with other elements of the
model, and how they can be managed in a structured and hierarchical environment


\end{block}
\end{frame}


\newpage
\begin{frame}
\frametitle{ System Acquisition and Management }
\begin{block}{Vendor presentation }

Foundational problems and needs
associated with  complex systems acquisition   {  All kind of challenges !!!!! }


\end{block}
\end{frame}



\newpage
\begin{frame}
\frametitle{ System Acquisition and Management }
\begin{block}{Complex Systems Acquisition }

requires a higher level of coupling between system
engineering and the acquisition process to support SoS ( system of systems ), as well as the need for higher
levels of lead system integrator (LSI) support

\end{block}
\end{frame}


\newpage
\begin{frame}
\frametitle{ System Acquisition and Management }
\begin{block}{Acquisition Timeliness  }

The systems that are
developed by the acquisition process need to have risk-managed processes that are not just
qualitative (the dominant method used today) but also quantitative where the risks can be
measured with accepted and agreed-upon metrics that can be tightly coupled and integrated
into an engineering system.

\end{block}
\end{frame}


\newpage
\begin{frame}
\frametitle{ System Acquisition and Management }
\begin{block}{Acquisition Process  }

The acquisition process is not design-driven. The  acquisition process is
oversight-driven and document-driven.

Cost optimization is very difficult to
achieve because system performance is hard to quantifiably measure as the system is
being developed and assessments and trade-offs are being made. Finally, there is limited
modeling of the system to vet lessons-learned which would enhance one’s ability to make
supportability improvements or forge an improvement strategy. 


\end{block}
\end{frame}


\newpage
\begin{frame}
\frametitle{ System Acquisition and Management }
\begin{block}{Total Ownership Costs  }

Total ownership costs (TOC) are difficult to predict and control. The acquisition cost
incurred during the development cycle is only a fraction of the total ownership cost of any
system. In fact, the development cost is often the minority cost component of the TOC of the
system throughout its lifetime.


\end{block}
\end{frame}

\newpage
\begin{frame}
\frametitle{ System Acquisition and Management }
\begin{block}{acquisition engine and associated engineering system }

how they map against the needs and requirements of the acquisition community, and
finally, what would be needed in addition to integrate these tools.


\end{block}
\end{frame}


\newpage
\begin{frame}
\frametitle{ System Acquisition and Management }
\begin{block}{SysML modeling tool }

 Input : SysML specification.
 
 ( use Gaphor and generate SysML file )
 
SysML is used to 
 generate a Python data model from a gaphor model

 Output : Python Data Model

 python codegen.py modelfile outfile overrides

  python codegen.py tmodemv1.gaphor  j1.py overrides

  python coder.py tmodemv1.gaphor  j1.py overrides

\end{block}
\end{frame}
