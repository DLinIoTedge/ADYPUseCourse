

\title[Systems Engineering]{Systems Complexity } 





\newpage
\begin{frame}
\frametitle{Ground Station for CubeSat Project }
\begin{block}{On Measuring complexity}

\textcolor{teal}{
Algorithmic entropy (also known as Kolmogorov complexity or complexity for short) evaluates the complexity of the description required for a lossless recreation of the network}


 
\end{block}
\end{frame}


\newpage
\begin{frame}
\frametitle{Ground Station for CubeSat Project }
\begin{block}{Requirements for CubeSat Ground Station}

\textcolor{teal}{
Mani is project manger of Cubesat in Bangalore, He had collected requirements from 4 different teams.   Anbu is ground station team and he said that " L band" link is good for uplink and downlinl. he need 5 MHz L-band spectrum. And also he need a office with infra in Hosur. Ground station office will have 15 people working and reporting to him. He said that, he can uplink and downlink to cubesat at one side and other side, he can connect with 4G wireless and also  FTTH  optical cable based internet.  }

 
\end{block}
\end{frame}


\newpage
\begin{frame}
\frametitle{Ground Station for CubeSat Project }
\begin{block}{ Team for CubeSat Ground Station}

 \textcolor{teal}{
 Lath is reporting to him and she is working on Human resource department. She need to get JDs  to get team.  But Mani had committed july 17 as opening day. Now it is June 13th . How Anbu  and latha will able to meet the above delivery date.   Does this requirements looks feasible?.  In case yes, how to encode it in Symbols such that  encoded symbols used to compute K(s).?}

\end{block}
\end{frame}


\newpage
\begin{frame}
\frametitle{Ground Station for CubeSat Project }
\begin{block}{ Time Line and Feasibility}

\textcolor{teal}{
To evaluate whether the given requirements are feasible within the provided timeline, let's break down the tasks and analyze the situation:}


\begin{enumerate}
    \item  Anbu's Requirements:
    
    \begin{enumerate}
        \item  \textcolor{teal}{ L-band spectrum requirement: Anbu needs 5 MHz of L-band spectrum for the uplink and downlink.}
        \item \textcolor{teal}{ Ground station office in Hosur: Anbu requires an office space with infrastructure to accommodate 15 people.}
    \end{enumerate}
    
    \item  Lath's Requirements:
       \begin{enumerate}
        \item  \textcolor{teal}{Job Descriptions (JDs): Lath needs to gather the job descriptions to form a team.}
     \end{enumerate}
    
\end{enumerate}

 
\end{block}
\end{frame}


\newpage
\begin{frame}
\frametitle{Ground Station for CubeSat Project }
\begin{block}{Study on Anbu's requirments }


\textcolor{red}{Considering the current date is June 13th and the commitment for the opening day is July 17th, there is approximately one month available to fulfill the requirements. Let's assess each requirement:}




 
\end{block}
\end{frame}


\newpage
\begin{frame}
\frametitle{Ground Station for CubeSat Project }
\begin{block}{Study on Anbu's requirments}

Anbu's requirements:


\textcolor{red}{1. L-band spectrum: Acquiring a 5 MHz L-band spectrum can depend on several factors such as licensing, regulatory approvals, and availability. It would be important for Anbu to initiate the necessary steps to obtain the required spectrum promptly. This process may involve coordination with regulatory bodies and potential negotiations with relevant authorities. Depending on the specific circumstances, this requirement might be achievable within the given timeline.}

 
\end{block}
\end{frame}



\newpage
\begin{frame}
\frametitle{Ground Station for CubeSat Project }
\begin{block}{Study on Anbu's requirments}

Anbu's requirements: 


\textcolor{red}{2. Ground station office: Securing an office space in Hosur capable of accommodating 15 people may involve searching for suitable locations, negotiating leases, and setting up the necessary infrastructure. This process typically takes time, including finding the right property, finalizing agreements, and ensuring all infrastructure requirements are met. Given the timeline constraint, it might be challenging to establish and set up the office by July 17th unless existing options are readily available.}



 
\end{block}
\end{frame}


\newpage
\begin{frame}
\frametitle{Ground Station for CubeSat Project }
\begin{block}{Study on Latha's requirments}



\textcolor{red!65}{Lath's requirements:}

\textcolor{red!65}{Job Descriptions: Gathering and finalizing job descriptions for the team can be a time-consuming task. It involves defining roles, responsibilities, and requirements for each position. Depending on the number of positions and the complexity of the roles, this task can take some time. However, if there are predefined templates or previous JDs available, it could expedite the process.}




 
\end{block}
\end{frame}


\newpage
\begin{frame}
\frametitle{Ground Station for CubeSat Project }
\begin{block}{ncode the requirements in symbols}


\textcolor{blue}{To encode the requirements in symbols for computing K(s), we need a clear understanding of the symbols and their significance in the given context. It seems that the provided information does not specify the exact symbols or the purpose of computing K(s). Hence, it is difficult to provide further guidance on this aspect.}





 
\end{block}
\end{frame}



\newpage
\begin{frame}
\frametitle{Ground Station for CubeSat Project }
\begin{block}{feasibility of their respective requirements}


\textcolor{blue!65}{In summary, meeting the given delivery date of July 17th might be challenging considering the requirements, time constraints, and tasks involved. It would be advisable for Anbu and Lath to assess the feasibility of their respective requirements, identify potential bottlenecks, and collaborate with the project manager, Mani, to prioritize tasks and explore alternative options to meet the project's objectives within the given timeframe.
 }




\end{block}
\end{frame}


\newpage
\begin{frame}
\frametitle{Ground Station for CubeSat Project }
\begin{block}{requirements for Ground Station  in to symbols }

\textcolor{teal}{
How to convert above requirements for Ground Station  in to symbols  }

To convert the given requirements for the Ground Station into symbols, we can represent the different aspects using appropriate symbols. Here's one way to represent the requirements using symbolic notation: 
 
\end{block}
\end{frame}

\newpage
\begin{frame}
\frametitle{Ground Station for CubeSat Project }
\begin{block}{requirements for Ground Station  in to symbols }

\begin{enumerate}
    \item  \textcolor{teal}{ L-band spectrum requirement: Symbol: L$\_$band,}
Spectrum requirement: 5 MHz, Representation: L$\_$band = 5 MHz
    \item \textcolor{teal}{ Office space requirement in Hosur: 
Symbol: Office$\_$Hosur,Representation: Office$\_$Hosur}
    \item \textcolor{teal}{ Number of people working in the ground station office:  Symbol: People$\_$count , Number of people: 15 , Representation: People$\_$count = 15}
    \item  \textcolor{teal}{Connectivity options for the ground station:
 Symbol: Connectivity , 
Options: 4G wireless, FTTH optical cable, 
Representation: Connectivity = {4G wireless, FTTH optical cable}}
\end{enumerate}

 
\end{block}
\end{frame}

\newpage
\begin{frame}
\frametitle{Ground Station for CubeSat Project }
\begin{block}{encoded symbols for the given requirements }

\textcolor{teal}{
Combining the symbols and representations, the encoded symbols for the given requirements would be:}

\begin{enumerate}
    \item Office$\_$Hosur
     \item   L$\_$band = 5 MHz
     \item  People$\_$count = 15
     \item  Connectivity = {4G wireless, FTTH optical cable}
\end{enumerate}

\textcolor{teal}{
These symbols can be used to represent and refer to the respective requirements of the Ground Station in a concise and standardized manner.}

\end{block}
\end{frame}


\newpage
\begin{frame}
\frametitle{Ground Station for CubeSat Project }
\begin{block}{Compute K(s)}

\textcolor{teal}{
For requirements, encoded symbols created, how to use them to compute K(s)  }

To use the encoded symbols to compute K(s), you need to define the mathematical relationship or formula that relates the encoded symbols. Since the specific purpose or mathematical relationship is not provided in the given information, I'll provide a generic example to illustrate how you can use the encoded symbols to compute a value, K(s).

\end{block}
\end{frame}


\newpage
\begin{frame}
\frametitle{Ground Station for CubeSat Project }
\begin{block}{encoded symbols}

Let's consider the following encoded symbols: 

\begin{enumerate}
    \item  L$\_$band = 5 MHz
  \item  Office$\_$Hosur
    \item  People$\_$count = 15
      \item  Connectivity = {4G wireless, FTTH optical cable}
\end{enumerate}

\end{block}
\end{frame}


\newpage
\begin{frame}
\frametitle{Ground Station for CubeSat Project }
\begin{block}{computation formula}

Now, let's define a simple example computation formula:

\[
K(s) = \frac{(Lband \times Peoplecount)}{ |Connectivity|} 
\]

\end{block}
\end{frame}


\newpage
\begin{frame}
\frametitle{Ground Station for CubeSat Project }
\begin{block}{computation formula}


Example, we are multiplying the L$\_$band value by the number of people (People$\_$count) and dividing it by the number of connectivity options (Connectivity represents the count of options).

Using the encoded symbols, we substitute the corresponding values into the formula:


 
\[ K(s) =  \frac{5 MHz \times 15}{2} \]

\end{block}
\end{frame}


\newpage
\begin{frame}
\frametitle{Ground Station for CubeSat Project }
\begin{block}{computation formula}

If we assume that MHz is a unit of frequency, we need to convert it to a numerical value. 

Let's consider 1 MHz = $10^6$ Hz.

\[ 
K(s) = \frac{(5 * 10^6 Hz * 15)}{  2}
\]

Performing the calculation, we get:
\[ 
K(s) = 37.5 * 10^6  \textrm{Hz} 
\]


\end{block}
\end{frame}


\newpage
\begin{frame}
\frametitle{Ground Station for CubeSat Project }
\begin{block}{real-world scenario}

\textcolor{teal}{
This is just a generic example to demonstrate how you can use the encoded symbols to compute a value, K(s). In a real-world scenario, the computation and relationship between the symbols would depend on the specific requirements and mathematical model associated with the project.}


\end{block}
\end{frame}



\newpage
\begin{frame}
\frametitle{Ground Station for CubeSat Project }
\begin{block}{formulation of the computation formula}

\textcolor{teal}{
To compute K(s) in your specific case, you would need to define the mathematical relationship or formula that represents the objective or purpose of the computation. The specific requirements and context would guide the selection and formulation of the computation formula.}

\end{block}
\end{frame}

\newpage
\begin{frame}
\frametitle{Ground Station for CubeSat Project }
\begin{block}{mathematical relationship formula }

\textcolor{teal}{
To define a more accurate and meaningful mathematical relationship formula for the encoded symbols, it is necessary to have a clear understanding of the desired outcome or metric to be computed based on the given requirements.}


\end{block}
\end{frame}



\newpage
\begin{frame}
\frametitle{Ground Station for CubeSat Project }
\begin{block}{Revise K(s)}

yes, requirements are given,  get K(s)

Based on the provided requirements, we can compute K(s) using the following formula:

$ K(s) = \frac{Lband  \times Peoplecount}{|Connectivity|}   $


\end{block}
\end{frame}



\newpage
\begin{frame}
\frametitle{Ground Station for CubeSat Project }
\begin{block}{ K(s) revsied }

Substituting the given values: 

\begin{enumerate}
    \item  L$\_$band = 5 MHz,
        \item  People$\_$count = 15,
        \item  Connectivity = {4G wireless, FTTH optical cable}
\end{enumerate}





\[ K(s) = \frac{5 MHz * 15}{2} \]

Now, let's calculate K(s):  $K(s) = \frac{75 MHz}{2}$ 

K(s) =  37.5 MHz.



\end{block}
\end{frame}




\newpage
\begin{frame}
\frametitle{Ground Station for CubeSat Project }
\begin{block}{K(s) not including when service will be ready}

above K(s) not including when service will be ready ( before july 17th is possible)

\[  K(s) = \frac{Lband \times Peoplecount}{|Connectivity|} \times (Readinessweight)   \]

For example, if we assume a Readiness$\_$weight of 0.5:

\[
K(s) = 37.5 MHz * 0.5 
\]

K(s) = 18.75 MHz


\end{block}
\end{frame}

\newpage
\begin{frame}
\frametitle{Ground Station for CubeSat Project }
\begin{block}{weight assigned }

\textcolor{teal}{
How 0.5 assigned ? }

Ultimately, the weight assigned should reflect the relative significance of the readiness aspect within the broader context of the project. It's important to involve relevant stakeholders and make informed decisions when assigning weights to different factors in the computation of K(s).

\end{block}
\end{frame}



\newpage
\begin{frame}
\frametitle{Ground Station for CubeSat Project }
\begin{block}{weight assigned }

Does 0.5 depends on Latha and Anbu?

Anbu need to get a team and make them ready work.  this is not included in computation of K(s)


Updated formula:
\[ 
K(s) = \frac{Lband \times Peoplecount}{|Connectivity|}    x Teamreadiness \] 

Team$\_$readiness , how to expand this in to computable parameter 

\end{block}
\end{frame}



\newpage
\begin{frame}
\frametitle{Ground Station for CubeSat Project }
\begin{block}{K(s) revised with Team Readiness}

For example, let's assume there are four criteria for Team$\_$readiness with equal weightage : 

Identify key factors that contribute to the readiness of Anbu's team

\begin{enumerate}
    \item  Completion of required training programs: Score = 8
    \item  Acquisition of necessary skills and certifications: Score = 7
    \item  Familiarity with relevant tools and technologies: Score = 9
    \item Clear understanding of roles and responsibilities: Score = 6

\end{enumerate}

Compute the Team$\_$readiness parameter:

\[ 
Team_readiness =\frac{(8 + 7 + 9 + 6)}{4}  \]

Team$\_$readiness =  7.5

\end{block}
\end{frame}


\newpage
\begin{frame}
\frametitle{Ground Station for CubeSat Project }
\begin{block}{HR Readiness}

Latha is not in K(s)

\[ 
K(s) = \frac{Lband \times Peoplecount}{|Connectivity|}   \times Teamreadiness \times  Hrreadiness
\] 

 Hr$\_$readiness is computable ?

\end{block}
\end{frame}



\newpage
\begin{frame}
\frametitle{Ground Station for CubeSat Project }
\begin{block}{Computability of HR Readiness}

For instance, let's assume there are three criteria for HR$\_$readiness with equal weightage:

\begin{enumerate}
    \item  Completion of recruitment and onboarding processes: Score = 7
\item  Availability of job descriptions (JDs): Score = 8
\item  Implementation of HR policies and procedures: Score = 6
\end{enumerate}



Compute the HR$\_$readiness parameter:

\[ HR_readinesss =\frac{(7 + 8 + 6)}{3}  \] 

 HR$\_$readiness =  7


Availability of people in Hosur will play role in  HR$\_$readiness

\end{block}
\end{frame}


\newpage
\begin{frame}
\frametitle{Ground Station for CubeSat Project }
\begin{block}{Computability of HR Readiness}

\[  
HR_readiness = \frac{AvailabilityofJDs + AvailabilityofPeopleinHosur}{2}
\]

Availability$\_$of$\_$People$\_$in$\_$Hosur is computable or not


 local talent pool is key before choose location for Ground Station for CubeSat

 K(s) = ( (Lband x PeopleCount) / Connectivity) x TeamReadiness x HRReadiness x AvailabilityOfPeopleInHosur

 K(s) = (Lband x PeopleCount) / (Connectivity x TeamReadiness x HRReadiness x AvailabilityOfPeopleInHosur x SpectrumAvailability)
 
\end{block}
\end{frame}


\newpage
\begin{frame}
\frametitle{Ground Station for CubeSat Project }
\begin{block}{Delay in K(s) }

Potential delay in obtaining the L-band spectrum : is computable

K(s) = (Lband x PeopleCount) / (Connectivity x TeamReadiness x HRReadiness xAvailabilityOfPeopleInHosur x DelayFactor)

\end{block}
\end{frame}


\newpage
\begin{frame}
\frametitle{Ground Station for CubeSat Project }
\begin{block}{DelayFactor}

\textcolor{teal}{
DelayFactor is computable? }

It is important to note that the accuracy and reliability of the DelayFactor will depend on the available data and the level of certainty associated with the potential delays in obtaining the L-band spectrum.

Bribe might be involved

As an AI language model, I must emphasize that I cannot support or provide guidance on engaging in any illegal or unethical activities, including bribery. 

DelayFactor is computable? 

\end{block}
\end{frame}


\newpage
\begin{frame}
\frametitle{Ground Station for CubeSat Project }
\begin{block}{mathematical relationship for DelayFactor }

\textcolor{teal}{
DelayFactor  in the form of mathematical relationship  }

DelayFactor = f(factors)

\textcolor{teal}{
In this representation, "factors" denote the specific elements or variables that contribute to the delay in obtaining the L-band spectrum. These factors can include regulatory processes, administrative procedures, coordination issues, or any other relevant considerations }


DelayFactor = (w1 x Factor1) + (w2 x Factor2) + ... + (wn x Factorn)


Factor1 = T1

Here, T1 represents the time taken for regulatory approval.

\end{block}
\end{frame}



\newpage
\begin{frame}
\frametitle{Ground Station for CubeSat Project }
\begin{block}{ time taken}

\textcolor{teal}{
How much  time taken?  }


time taken in  regulatory approval process

so time taken is not computable?

\textcolor{teal}{
To incorporate the time taken into a computational model or analysis, you can consider using probabilistic or statistical approaches. By collecting historical data, assessing past approval times, and accounting for potential variations, you can establish probability distributions or ranges that reflect the likely duration of the regulatory approval process.}

\end{block}
\end{frame}


\newpage
\begin{frame}
\frametitle{Ground Station for CubeSat Project }
\begin{block}{estimations, probabilities,}

\textcolor{teal}{
Therefore, while the exact time taken may not be computable with certainty, you can still work with estimations, probabilities, or ranges to account for the variability and uncertainty associated with the regulatory approval process.}


\end{block}
\end{frame}


\newpage
\begin{frame}
\frametitle{Ground Station for CubeSat Project }
\begin{block}{time taken  should not be in probabilistic}

\textcolor{teal}{
Digital india , time taken  should not be in probabilistic  and it should be deterministic  }


in dubai or signapore

\textcolor{teal}{
In locations such as Dubai or Singapore, which are known for their efficient and streamlined regulatory environments, the time taken for regulatory approval processes can generally be more deterministic compared to other regions. These countries have established frameworks and processes aimed at facilitating business activities and minimizing administrative hurdles.}


\end{block}
\end{frame}



\newpage
\begin{frame}
\frametitle{Ground Station for CubeSat Project }
\begin{block}{Factor2}


Factor2: Stakeholder Coordination and Negotiation

why Negotiation, just pay for spectrum

K(s) = LbandAvailability x SpectrumAvailability x TeamReadiness x HRReadiness x AvailabilityOfPeopleInHosur x DelayFactor

K(s) with DelayFactor  , may be with Bayes theorem on DelayFactor 

\end{block}
\end{frame}


\newpage
\begin{frame}
\frametitle{Ground Station for CubeSat Project }
\begin{block}{Revised K(s)}

K(s) = LbandAvailability x SpectrumAvailability x TeamReadiness x HRReadiness x AvailabilityOfPeopleInHosur x P(Delay | DelayFactor)


\end{block}
\end{frame}


\newpage
\begin{frame}
\frametitle{Ground Station for CubeSat Project }
\begin{block}{NN in K(s)}


Use NN to get P(Delay | DelayFactor)

\textcolor{teal}{
Using a neural network (NN) to estimate P(Delay | DelayFactor) is certainly a valid approach. Neural networks can be trained to learn complex patterns and relationships from data, making them suitable for probabilistic modeling tasks.}

Estimated probability of delay: 0.4074404537677765

 depends on the availability of relevant data 
 

\end{block}
\end{frame}





\newpage
\begin{frame}
\frametitle{Systems Complexity }
\begin{block}{On Measuring  Complexity of Systems }



\begin{figure}[hbt!]
\begin{tikzpicture}[
roundnode/.style={circle, draw=green!60, fill=green!5, very thick, minimum size=7mm},
squarednode/.style={rectangle, draw=red!60, fill=red!5, very thick, minimum size=5mm},
s1/.style={rectangle, draw=cyan!60, fill=red!5, very thick, minimum size=5mm},
s2/.style={rectangle, draw=cyan , fill=cyan!5, very thick, minimum size=5mm},
s3/.style={rectangle, draw=green , fill=green!5, very thick, minimum size=5mm},
s4/.style={rectangle, draw=gray , fill=gray!5, very thick, minimum size=5mm},
]

\tikzstyle{block} = [draw=green, fill=green!15, rectangle, text width = 5.5em, 
  text centered, minimum height = 6mm, node distance = 5em];

\tikzstyle{block1} = [draw=yellow, fill=yellow!15, rectangle, text width = 5.4em, 
  text centered, minimum height = 6mm, node distance = 5em];
  
  
%Nodes
\node[squarednode]      (maintopic)                              {Design};
\node[roundnode]        (uppercircle)       [draw =red, fill=red!23, above=of maintopic] { \textcolor{Red}{Fund}};
\node[squarednode]      (rightsquare)       [right=of maintopic] {Make};
\node[roundnode]        (inference)       [below=of rightsquare] {Deploy};
\node[squarednode]      (dataset)            [left=of maintopic] {Reuirements};
\node[block]      (students)            [above=of dataset, font=\tiny, yshift=-1.4cm] {\textcolor{purple}{Ground Station  in Hosur for CubeSat }   };

\node[block1]      (DLtrain)            [right=of uppercircle , font=\tiny, xshift = -1.4cm] {
\textcolor{purple}{Loan with \@ 16 $\%$ interest or Equity or Grants} };
\node[block1]      (cpu)            [above=of rightsquare , yshift =-1.6cm , font=\tiny] {\textcolor{purple}{India,China,West} };
\node[block1]      (tcpu)            [below=of maintopic ,  font=\tiny, 
 yshift=1.3cm] { \textcolor{gray!45}{Local design }  or 
\textcolor{red}{buy}};
\node[block1]      (ks)            [below=of dataset , font=\tiny, yshift=1.3cm] { \textcolor{gray!45}{Choose UTM }   \textcolor{gray!45}{and Compute}  \textcolor{green}{K(s)}};



%Lines
\draw[->] [color =blue!35] (uppercircle.south) -- (maintopic.north);
\draw[->] [color =blue!35] (maintopic.east) -- (rightsquare.west);
\draw[->] [color =blue!35] (rightsquare.south) --  (inference.north);

\draw[->] [color =blue!35] (students.south) -- (dataset.north);
\draw[->] [color =pink] (DLtrain.west) -- (uppercircle.east);
\draw[->] [color =pink] (cpu.south) -- (rightsquare.north);
\draw[->] [color =pink] (tcpu.north) -- (maintopic.south);

\draw[->] [color =pink] (dataset.south) -- (ks.north);

\draw[->] [color =pink] (ks.east) -- (tcpu.west);

\end{tikzpicture}
\caption{ Work Flow in  System Engineering }
\label{overviewwrkflowinAIapplicaion}
\end{figure}
 
\end{block}
\end{frame}
