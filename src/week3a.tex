
\title[Systems Engineering]{Security} 




\newpage 
\begin{frame}
\frametitle{Model-Based Engineering for Cybersecurity}

\begin{block}{ISO/SAE-21434  }
Engineering projects requires to consider the increasingly significant needs and constraints regarding expected behaviors, services, quality and security. 
\end{block}
\end{frame}


\newpage 
\begin{frame}
\frametitle{Model-Based Engineering for Cybersecurity}

\begin{block}{ISO/SAE-21434  }
These requirements are introduced into system and software engineering projects as functional and non-functional properties. Satisfying such properties implies rigorous processes that steer the project, from the requirements identification and definition to the system deployment and maintenance. 
\end{block}
\end{frame}



\newpage 

\begin{frame}
\frametitle{Model-Based Engineering for Cybersecurity}

\begin{block}{DLtrain }

%\url{https://www.iso.org/obp/ui/#iso:std:iso-sae:21434:ed1:v1:en}  
%https://www.iso.org/obp/ui/#iso:std:iso-sae:21434:ed1:v1:en 
\colorbox{red}{ \textcolor{white}{ For Consumer Automotive Industry ( non AI ) } }

DO-326 is the aviation counterpart of the automotive ISO/SAE 21434 standard
\colorbox{red}{ \textcolor{white}{ For  Aerospace  Industry ( non AI ) } }

\url{https://www.jkuse.com/dltrain/deploy-dl-networks}
\colorbox{green}{ \textcolor{white}{ For Consumer Automotive  and Aerospace Industry ( World of AI and security ) } }

\end{block}
\end{frame}



\newpage 

\begin{frame}
\frametitle{Model-Based Engineering for Cybersecurity}

\begin{block}{SAP}
Systems Engineering in PLM and SAP System Landscapes
SAP Intelligent Product Design  an important step on the way to the Single Source of Truth

\url{}https://www.cenit.com/en_EN/newsroom/article/news/systems-engineering-in-plm-and-sap-system-landscapes-783.html

https://help.sap.com/doc/3ba53e057ee8479b860fec1d46818b95/Cloud/en-US/Feature_Scope_Description.pdf

https://help.sap.com/docs/PLM_CP?locale=en-US

https://help.sap.com/docs/PLM_CP?locale=en-US

Systems can be developed according to the RFLP approach (Requirements engineering, Functional design, Logical design and Physical design).



\end{block}
\end{frame}


\newpage 

\begin{frame}
\frametitle{Model-Based Engineering for Cybersecurity}

\begin{block}{MBSE }


Model-Based System Engineering (MBSE) is an effective approach to address security requirements and risk assessment at the early stages of the development life cycle, which enables cost-efficient fixes. 


\end{block}
\end{frame}


\newpage 

\begin{frame}
\frametitle{Model-Based Engineering for Cybersecurity}

\begin{block}{MBSE }


 How cybersecurity risk assessment could be integrated into model-based requirement engineering. 


\colorbox{red}{ \textcolor{white}{ Solution } }

Model-based Cyberisk Assessment (MBCA) method, 

\begin{enumerate}
    \item  A semantic alignment between risk assessment concepts ( quantitative and qualitative ) and system modeling concepts ( statistical, stochastic , Dynamical continuous,  AI NN, AI CNN etc ).
    \item A \colorbox{green}{ \textcolor{white}{  modeling language extension } } to represent security concepts and metrics throughout the system modeling life cycle.
\end{enumerate}

\url{https://ieeexplore.ieee.org/document/9604634}

 Application :  industrial in-flight entertainment system.
 
\end{block}
\end{frame}


\newpage 

\begin{frame}
\frametitle{Model-Based Engineering for Cybersecurity}

\begin{block}{Understanding of CPSs  }
Cyber-physical systems (CPS), such as autonomous vehicles, are intelligent and networked. Close collaboration between stakeholders from different disciplines is necessary right from the start of development. 
\end{block}
\end{frame}


\newpage 

\begin{frame}
\frametitle{Model-Based Engineering for Cybersecurity}

\begin{block}{Understanding of CPSs  }
 In the automotive sector in particular, the collaboration of the car manufacturer extends to several suppliers. The increasing complexity in the design of such CPSs makes interdisciplinary and cross-company collaboration more difficult. Here, requirements specifications serve as a support for communication. 
\end{block}
\end{frame}

\newpage 

\begin{frame}
\frametitle{Model-Based Engineering for Cybersecurity}

\begin{block}{Problem in ISO/SAE 21434  }
A lack of overall understanding of such CPSs and their numerous interfaces jeopardizes the assurance of safety-relevant security. ISO/SAE 21434, which applies to the automotive industry, requires the creation of a cybersecurity concept at the beginning of the product development process. 

The problem is that ISO/SAE 21434 only prescribes WHAT must be done, but does not define 

\colorbox{red}{ \textcolor{white}{HOW this is supposed to be done methodically.} }

\end{block}
\end{frame}

\newpage 

\begin{frame}
\frametitle{Model-Based Engineering for Cybersecurity}

\begin{block}{Understanding of CPSs  }
Existing methods are not applicable to the concept phase without extensive tailoring .

\end{block}
\end{frame}


\newpage 

\begin{frame}
\frametitle{Model-Based Engineering for Embedded System Security}

\begin{block}{Embedded System Security }

\begin{enumerate}
    \item Build security into (product)  system
     \item  Verify the effectiveness of your security architecture
 \item  Identify potential vulnerabilities early in the software development life cycle  ( Open source vs Proprietary software )
  \item  Use analytic methods to increase the confidence in your design
   ( statistical methods vs  Stochastic Process vs Deep Learning AI Methods)
 \item  Develop updates in response to new threats  ( kanshi ) 
 \url{https://github.com/DLinIoTedge/dltrainBook/tree/jk/Edge/Kanshi }
\end{enumerate}

\end{block}
\end{frame}


\newpage 

\begin{frame}
\frametitle{Model-Based Engineering for Embedded System Security}

\begin{block}{Compliance to Industry Standards }

Automate verification of models and code to fulfill requirements from cybersecurity standards such as ISO/SAE 21434, IEC-26443, and DO-326. The IEC Certification Kit provides an overview on how to apply MATLAB, Simulink, Polyspace, and add-on products to ISO/SAE 21434, in addition to functional safety standards (ISO26262, IEC 61508). The kit helps you to build and qualify your development process for safe and secure embedded systems.

\url{https://in.mathworks.com/solutions/embedded-security.html}

\end{block}
\end{frame}

\newpage 

\begin{frame}
\frametitle{Model-Based Engineering for Embedded System Security}

\begin{block}{Add Countermeasures }

Prevent and mitigate potential vulnerabilities with robust design, state-of-the-art detection mechanisms, and security controls. Learn how to apply machine learning to implement an intrusion detection system (IDS) against spoofing. Leverage neural networks to enhance the robustness of image classification algorithms against adversarial attacks.


\url{ https://in.mathworks.com/help/comm/ug/design-a-deep-neural-network-with-simulated-data-to-detect-wlan-router-impersonation.html}

but use Kanshi ( which is free )

\end{block}
\end{frame}




\newpage 

\begin{frame}
\frametitle{Model-Based Engineering for Embedded System Security}
\begin{block}{Embedded System Security }

\url{https://in.mathworks.com/solutions/embedded-security.html}  Root folder

\url{https://in.mathworks.com/products/polyspace.html}  Polyspace
Making Critical Code Safe and Secure

\end{block}
\end{frame}

\newpage 

\begin{frame}
\frametitle{Model-Based Engineering in SAP EPD}
\begin{block}{SAP EPD}



\begin{enumerate}
\item  Open the link:   
\url{https://sap.sharepoint.com/:u:/s/102012/ESSCguWhTU9GjdYViBOfjVIBU4GyBPx_N-A9H4X8_afhvA}

\item  Download the linked ReqIF file to your local PC and remember the local folder for the next step.



\item  On the overview page choose Browse Repository under Quick Links.

\item  Choose the Test folder.

\item Select the Tools button in the upper right screen area.



\item  Choose Create new Folder

\item  In the dialog box enter a Name (e.g. "Valve System <today´s date> - XY" - for XY enter a personal identification like your initials) and choose Create.


\end{enumerate}
\end{block}
\end{frame}


\newpage 

\begin{frame}
\frametitle{Model-Based Engineering in SAP EPD}
\begin{block}{SAP EPD}

\begin{enumerate}


\item  Select the Tools button again and choose Import ReqIF File.

\item  Select Browse and choose the ReqIF file you just downloaded.
\item  Select Import.
\item  Choose Confirm.
\item  In the middle of the screen, in the section Diagrams you can open the imported Requirements List (Valve Requirements) by clicking on it
  
\item  Choose Edit in the upper right corner to add new requirements to the list.
\item  Choose the icon Insert Requirement in the header tool bar and enter a Title (e.g. Performance Characteristics) on the right side of the screen.
\item The requirement has been added at the top level.

Select your requirement and choose Insert Child Requirement in the header tool bar.

On the right side of screen, enter the title of the child requirement (e.g. Pressure).

Optionally you can add further details in the details of the info panel on the right.
\item  Choose Save.
\item  In the upper right corner choose Publish to make it available to your colleagues.
\item  Optionally add a comment in the dialog box and choose OK.
\item 
\item 
  
\end{enumerate}
\end{block}
\end{frame}



\newpage 

\begin{frame}
\frametitle{Model-Based Engineering in SAP EPD}
\begin{block}{SAP EPD}

\begin{enumerate}

\item https://www.sap.com/products/scm/enterprise-product-development.html?btp=3fdec6d4-aef3-45ec-98d2-b76fb20e8c3c  trial

 	
Who do I contact if I need help with my cloud trial?

Please email us at saptrialsupport@sap.com with any help or support questions about your trial.
\end{enumerate}
\end{block}
\end{frame}


\newpage 

\begin{frame}
\frametitle{Model-Based Engineering  }
\begin{block}{SE and Data science}

\url{https://towardsdatascience.com/machine-learning-a-systems-engineering-perspective-1be9d13040e7}

\url{https://www.ibm.com/products/architect-for-systems-engineers/details} IBM solution   IBM Engineering Systems Design Rhapsody – Architect for Systems Engineers

\url{https://github.com/mbe-shipyard/sysml.py}

\end{block}
\end{frame}


